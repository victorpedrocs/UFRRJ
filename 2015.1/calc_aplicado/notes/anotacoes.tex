\documentclass[portugues, 11pt]{article}
\usepackage{notes}
\usepackage{amssymb}
\usepackage{pgfplots}
\usepackage{amsmath}


% Uncomment these for a different family of fonts
% \usepackage{cmbright}
% \renewcommand{\sfdefault}{cmss}
% \renewcommand{\familydefault}{\sfdefault}

\newcommand{\thiscoursecode}{Cálculo Aplicado}
\newcommand{\thiscoursename}{TM406}
\newcommand{\thisprof}{Aleksandro}
\newcommand{\me}{Victor Pedro}
\newcommand{\thisterm}{2015.1}
\newcommand{\website}{google.com/+VictorPedro}
% Headers
\chead{\thiscoursename \ Notas de Aula}
\lhead{\thisterm}
\pgfplotsset{width=10cm, compat=1.9}


%%%%%% TITLE %%%%%%
\newcommand{\notefront} {
\pagenumbering{roman}
\begin{center}

{\ttfamily \url{\website}} {\small}

\textbf{\Huge{\noun{\thiscoursecode}}}{\Huge \par}

{\large{\noun{\thiscoursename}}}\\ \vspace{0.1in}

  {\noun \thisprof} \ $\bullet$ \ {\noun \thisterm} \ $\bullet$ \ {\noun {UFRRJ}} \\

  \end{center}
  }

% Begin Document
\begin{document}

  % Notes front
  \notefront
  % Table of Contents and List of Figures
  \tocandfigures
  % Abstract
  \doabstract{Este documento apresenta notas sobre a cadeira de Cálculo Aplicado no curso de Ciência da Computação.}
  
  	\section{Organização do Curso}
	\begin{align*}
		Unidade\, & 0 : Derivadas Parciais \\
				& 1 : Integrais Múltiplas \\
				& 2 : Tópicos de cálculo vetorial\\
	\end{align*}
	

	\section{Derivadas Parciais}
	
	Seja $f: D(f) \subset \mathbf{R}^n \rightarrow \mathbb{R}$ uma função de $n$ varáveis. O conceito de derivada parcial segue do fato de tratarmos de uma função de $n$ variáveis como uma função de uma variável considerando as demais fixas (constantes).
	
	\begin{defn}[n=2]
		Seja $f: D(f) \subset \mathbb{R}^2 \rightarrow \mathbb{R}$ uma função de duas variáveis $x$ e $y$. \\
		\begin{itemize}
			\item[(a)] a derivada parcial de $f$ em relação à $x$ é a função: $$ D_1f: D(f) \rightarrow \mathbb{R} $$ definida por
			$$ D_1 f(x,y) = \lim_{\Delta x \rightarrow 0} \frac{f(x + \Delta x, y) - f(x,y)}{\Delta x}$$.
			Se este existir.
			
			\item[(b)] Analogamente, a derivada parcial de $f$ em relação à $y$ é a função $ D_2 f: D(f) \rightarrow \mathbb{R} $ definida por:
			$$ D_2 f(x,y) = \lim_{\Delta y \rightarrow 0} \frac{f(x,y + \Delta y) - f(x,y)}{\Delta y} $$
			Se este existir.
		\end{itemize}
		
		\textbf{OBS.:} outras notações: $$ D_1 f, fx, f1, \frac{\delta f}{\delta x} $$
	\end{defn}
	
	\begin{exmp}
		Dada $f(x,y) = x^2-2xy+y^2$, encontre $D_1 f(x,y)$ e $D_2 f(x,y)$ utilizando a definição anterior.
		\begin{align*}
		D_1 f(x,y) & = \lim_{\Delta x \rightarrow 0} \frac{(x + \Delta x, y) - f(x,y)}{\Delta x} \\
		& = \lim_{\Delta x \rightarrow 0}
		\end{align*}
	\end{exmp}


	
	
  	\section{Exercícios Sobre Integral Dupla}
	
    	\begin{enumerate}
    		\item Encontre o volume do sólido limitado pela superfície $ f(f,x)=4 - \frac{x^2}{4} - \frac{y^2}{16}$, os planos $x=3$, $y=2$ e os planos coordenados.
    		\textbf{Resposta}
    		
    		\begin{equation}
    			V = \int_{R}\int{f(x,y) dA}
    		\end{equation}
    		
    		\begin{equation}
    			= \int_0^3\int_0^2{4-\frac{x^2}{9}-\frac{y^2}{16}} dy dx
    		\end{equation}
    		
    		\begin{equation}
    			= \int_0^3{4y-\frac{x^2y}{9}-\frac{y^3}{48}\mid_0^2}dx
    		\end{equation}
    		\begin{equation}
    			= \int_0^3{8-\frac{2x^2}{9}-\frac{8}{48}-(0-0-0)} dx
    		\end{equation}
    		\begin{equation}
    			= \int_0^3{\frac{2x^2}{9}+\frac{47}{6}}dx
    		\end{equation}
    		\begin{equation}
    			= \frac{2x^3}{27}+\frac{47x}{6}\mid_0^3
    		\end{equation}
    		\begin{equation}
    			= -2 + \frac{47}{2} - 0 = \frac{43}{2}
    		\end{equation}
    	\end{enumerate}
	
	%%%%%%%%%%%%%%%%%%%%%%%%%%%%
	\section{Teorema de Fubini}
	Seja $R$ o retângulo $R=[a.b]$ e $[c,d]$. Se $f$ é contínua em $R$, então
	\begin{equation}
		\int_R\int{f(x,y)} dA = \int_c^d\int_a^b{f(x,y)}dxdy 
		= \int_a^b\int_c^d{f(x,y)} dydx
	\end{equation}
	
	\underline{Ex.} Calcule $\int_R\int{y^2x}dA$ no retângulo $-3\leq x\leq 2$ e $0 \leq y \leq 1$.
	
	
	%%%%%%%%%%%%%%%%%%%%%%%%%%%%
	\section{Integrais Duplas em Regiões não-retangulares}
	As Integrais iteradas podem apresentar limites de integração não-ctes, como
	
	\begin{equation}
		\int_a^b\int_{g_{1}(x)}^{g_{2}(x)}{f(x,y) dydx}
	\end{equation}
	E
	\begin{equation}
		\int_c^d\int_{h_{1}(x)}^{h_{2}(x)}{f(x,y) dydx}
	\end{equation}
	
	%%%%%%%%%%%%%%%%%%%%%%%%%%%%
	
	%%%%%%%%%%%%%%%%%%%%%%%%%%%%
	\section{Integrais Triplas}
	A Integral tripla de uma função $f(x,y,z)$ é definida de forma análoga a integral dupla, nesse caso, $f(x,y,z)$ deve ser contínua em um sólido $G$ do espaço 3D.

	As propriedades de integral tripla são análogas às de integral dupla.

	Integrais triplas em "caixas" retangulares:

	\begin{thrm}
		Seja $G$ uma caixa retangular definida por $a\leq x \leq b$, $c \leq y \leq d$ e $l \leq z \leq m$, ou seja, 
		\begin{equation}
			G = [a,b]*[c,d]*[l,m].
		\end{equation}
		Se $f$ é uma função em $G$, então:
		\begin{equation}
			\iiint_G f(x,y,z) \,dV  \\
			= \int_l^m\int_c^d\int_a^b f(x,y,z) \,dx\,dy\,dz
		\end{equation}
		não importando a ordem de integração
	\end{thrm}

	\begin{exmp}
		Calcule $ \iiint_G {12xy^2z^3}{dv}$, onde $G$ é a caixa $-1 \leq x \leq 2$, $0 \leq y \leq 3$ e $0 \leq z \leq 2$.
		
		\textbf{Solução.}
		\begin{align*}
			\iiint_G 12xy^2z^3 \,dV \\
			= \int_-1^2\int_0^3\int_0^2 \,dz\,dy\,dx \\
			\textbf{\dots}
		\end{align*}
		

	\end{exmp}
	
	\subsection{Integrais triplas em Sólidos mais gerais}
	
	\textbf{(1) $G$ é um sólido do tipo $xy$ simples}
	\begin{thrm}
		Seja $G$ um sólido $xy$ simples com superfície inferior $z=g_1(x,y)$ e superfície superior $z=g_2(x,y)$. Seja $R$ a projeção de um $G$ no plano $xy$. Se $f$ é uma função contínua em $G$, então.
		\begin{align*}
		\iiint_G & {f(x,y,z)}{dv}
		\\
		\iint_G  & \left [\int_{g_1 (x,y)}^{g_2 (x,y)} {f(x,y,z)}{dz} \right ] {dA}
		\end{align*}
	\end{thrm}
	
	\textbf{(2) $G$ é um sólido do tipo $xz$ simples: }
	\begin{thrm}
		Seja $G$ um sólido $xy$ simples, limitado à esquerda pela superfície $y=a_1 (x,y)$ e pela direita pela superfície $y=g_2 (x,y)$. Seja $\underline{R}$ A projeção de $G$ no plano $xy$ se $f$ é contínua em $G$, então: 
		\begin{align*}
			\iiint_G & {f(x,y,z)} \,dv \\
			= \iint_R & \left[ \int_{g_1(x,z)}^{g_2(x,z)}	f(x,y,z) \,dy \right ] \,dA 
		\end{align*}
	\end{thrm}


	\textbf{(3) $G$ É um sólido $xy$ simples:}
	\begin{thrm}
		Seja $G$ um sólido $xy$ simples limitado atrás pela superfície $x=g_1(x,z)$, à frente pela superfície $x=g_2(x,z)$.
		Seja $\underline{R}$ a projeção de $G$ no plano $xz$. Se $f$ é contínua em $G$, então:
		
		\begin{align*}
			\iiint_G & f(x,y,z) \,dv
			= \iint_R \left [ \int_{g_1(x,z)}^{g_2(x,z)} f(x,y,z) \,dx \right ] \,dA
		\end{align*}
	\end{thrm}
	
	\begin{exmp}
		Seja $G$ a cunha do primeiro octante seccionada do sólido cilíndrico $y^2+z^2 \leq 1$, pelos planos $=y=x$ e $x=0$.
		Calcule $\iiint_G z \,dV$ \\
		\begin{figure}[h]
			\centering
			\includegraphics[width=0.8\textwidth]{imagens/solucao_integral_tripla_1}
			\caption{Solução - Parte 1}
			\label{fig:int_tripla_1}			
			\includegraphics[width=0.8\textwidth]{imagens/solucao_integral_tripla_2}
			\caption{Solução - Parte 2}
			\label{fig:int_tripla_2}			
		\end{figure}
	\end{exmp}
	
	\textbf{OBS} Podemos utilizar uma integral tripla para determinar o volume de um sólido $G$ da seguinte maneira:
	$$ V(G) = \iiint_G 1 \,dV$$
	
	\begin{exmp}
		Demonstre através de uma integral tripla que o volume do cilindro circular reto, de altura $H$ e raio da base $R$ é.
		$$V = \pi R^2H$$
		\underline{Sólido $xy$ simples}
		
		\begin{align*}
			\underline{região polar}& \,\,\,& 0 \leq z \leq H \\
			0 \leq \theta \leq H	& \,\,\, & V(G) = \iiint_G 1 \,dV \\
			0 \leq r \leq R 		& \,\,\, & = \iint_R \left [ \int_0^H 1 \,dV \right ] \,dA \\ 
		\end{align*}
	\end{exmp}
	
	\begin{exmp}
		Idem ao anterior, mas usando integral dupla:
		$$V(G) = \iint_R H \,dA = \dots = \pi R^2H.$$
	\end{exmp}
	
	\section{Coordenadas Ciliíndricas}
	
	As coordenadas cilíndricas $(r,\theta,z)$ de um ponto $(x,y,z)$ são definidas por:
	\begin{equation}
	\left \{
		\begin{tabular}{l}
			x = r $\cos \theta$ \\
			y = r $\sin \theta$ \\
			z = z
		\end{tabular}
	\right \}
	\end{equation}
	
	Onde $r\leq 0$ e $\theta$ é um ângulo que pode variar de zero à $2\pi$,insto é, $\theta \in [0,2\pi]$
	
	\textbf{OBS:} Com as igualdades acima, temos:
	\begin{itemize}
	\item $x^2 + y^2 r^2 \cos^2 \theta + r^2 \sin^2 \theta = r^2(\cos^2 \theta + \sin^2 \theta) = r^2. $\\
	\item $\tan \theta$
	\end{itemize}
	

	
	


		
	



	
	
	

	
  %%%%%%%%%%%%%%%%%%%%%%%%%%%%%%%%%%%%%%%%%%%%%%%
  \end{document}
